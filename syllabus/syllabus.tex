\documentclass[a4paper,12pt]{article}
\usepackage{hyperref,kpfonts,inconsolata,microtype,booktabs,enumitem,longtable,graphicx}
\hypersetup{colorlinks=true,urlcolor=blue}
\makeatletter\renewcommand\section{\@startsection {section}{1}{\z@}%
		    {-3.5ex \@plus -1ex \@minus -.2ex}%
		    {2.3ex \@plus.2ex}%
		    {\normalfont\large\scshape}}
\renewcommand\subsection{\@startsection{subsection}{2}{\z@}%
		{-3.25ex\@plus -1ex \@minus -.2ex}%
		{1.5ex \@plus .2ex}%
		{\normalfont\scshape}}
\renewcommand\thesection{\arabic{section} / }
\makeatother
\parindent 0pt
\parskip 7pt

\title{
    \includegraphics{rug}\\
    \textsc{Syllabus:} Coding for Humanities}
\author{Andreas van Cranenburgh}
\date{2020/2021}

\begin{document}
\maketitle
\thispagestyle{empty}
\pagestyle{empty}


\begin{tabular}{l p{0.75\textwidth} }
    Title:      & Coding for Humanities \\
    Code:       & LHU002M05 \\
    Programme:  & Master Communication and Information Studies (CIS) \\
    Tracks:     & Digital Humanities, Computer Communication \\
    Period:     & Semester 1a \\
    Type:       & Lectures and practical lab sessions \\
    Teacher:    & Andreas van Cranenburgh \\
    Credits:    & 5 ECTS \\
    Lectures:   & Monday 15:00-17:00, online \\
    Labs:       & Tuesday 15:00-18:00, online \\
\end{tabular}



\section{Type of course unit, number of ects credit points,
        and admission requirements}
\begin{enumerate}[label={(\alph*)}]
    \item Type: mandatory (except for students CIS-Computer Communication
        starting in February).
    \item Credit points: 5 ECTS (140 hours).
    \item Admission requirements: (International) Students admitted to the
        tracks Digital Humanities (DH) or Computer Communication (CC) of the
        Master CIS are allowed to follow this course. No knowledge of
        programming or computer science is assumed. In case the student has
        already followed programming courses, it is wise to consult the
        teachers or study advisor to see whether the course is worthwhile to
        follow.
\end{enumerate}

\section{Content of the course unit}

The course is meant to be an introduction to programming in view of using
technology in solving problems in the interdisciplinary fields of Digital
Humanities (DH) and Computer-Mediated Communication (CMC). The basics of
programming are introduced, as well as tools and libraries
that are useful for DH and CMC. The programming language Python
(version 3) is used for writing programs. Jupyter is used as environment to
develop and execute Python programs. More specifically, the following basic
programming topics are addressed in the course: variables, values of various
types, strings, lists, dictionaries, conditionals, iteration and looping,
files, and functions. This is followed by practical applications related to
text analysis and processing of numerical data contained in tables.

\section{Position of the course unit in the degree program}
The course unit is accessible to master students of Communication and
Information Studies (tracks Digital Humanities and Computer Communication).
The course introduces programming concepts and techniques, and practices
programming skills which are necessary or useful for subsequent courses in the
programmes of Digital Humanities and Computer Communication.

\section{Learning outcomes of the course unit}

Upon successful completion of the course unit, students are able to:

\begin{enumerate}
    \item Write simple programs to perform basic tasks such as searching and
        cleaning text corpora
        (Application of Knowledge and Insight).
    \item Work with Jupyter Notebooks and other common Python data science
        tools to report on simple exploratory experiments: load a tabular
        dataset, compute summary statistics, and create plots
        (Application of Knowledge and Insight).
    \item Understand and solve common errors during programming
        (Application of Knowledge and Insight).
    \item Read documentation on available software to evaluate its
        applicability to a problem
        (Learning skills).
    \item Collaborate effectively with programmers using proper terminology
        (Communication).
\end{enumerate}

\section{Mode of instruction and learning activities}
The course is based on the principles of active learning. In weekly lectures,
new programming concepts and techniques are introduced. Self-study is
required to get a better, deeper, and more nuanced understanding of the newly
introduced subject material. In the labs, students apply the knowledge obtained
in simple exercises, discussion, and programming assignments. Students have to
hand in two substantive and complex assignments (take-home exams), which will
be assessed (see Weekly Timetable in section 10). % ~\ref{sectimetable}).

% Computers in the lab have the necessary software installed, but it may be
% easier for you to use your own laptop. If you have one, please bring it to
% class.

Due to the current plague situation, all lectures and labs will be held
online. The course is primarily taught synchronously, i.e., online lectures
and labs with live interaction held at a specify weekly time slot.

However, asynchronous means are also offered: the lectures will be recorded and
available to re-watch at a later time, and a discussion board on Nestor gives
the opportunity to ask practical questions outside lab hours. The labs will
be recorded partially; demonstrations and discussions of solutions to
exercises in the first part of the lab by the teacher or teaching assistant
may be recorded, but naturally, none of the 1-on-1 interactions in the rest of
the lab will be recorded.

Attendance at the online lectures and labs is encouraged since this allows for
the possibility to ask questions and other ways of interaction.

\section{Assessment}
The final course grade is based on the two take-home exams due in week 4 and
week 8. Both exams have to be sufficient (at least 5.5) in order to pass
the course. The final score is the mean of the grades for the two exams,
weighted equally (50~\%). There is one re-sit opportunity for each exam, which
is available in case of an insufficient grade. Deadlines for handing in
exams are given in the weekly timetable, and on Nestor.

\section{Cheating and plagiarism}
Cheating and plagiarism are subject to the provisions set down in the TER
(Article 8.17 of Part A of the BA TER or Article 4.13 of Part A of the MA TER).
The Board of Examiners is always informed in cases of suspected cheating or
plagiarism.

\section{Calculation of student workload}
\begin{tabular}{ll}
    Participation lectures, labs   & 35 hours \\
    Literature to be studied       & about 39 hours \\
    Assignments                    & about 50 hours \\
    Take-home exams                & about 16 hours \\
    \emph{Total}                   & about 140 hours (5 ECTS) \\
\end{tabular}

\section{Literature}
Severance, C. (2016). \emph{Python for everybody}.
Available online: \url{https://www.py4e.com/book.php}

Karsdorp, F. and M. van Gompel. \emph{Python Programming for the Humanities}.
\url{http://www.karsdorp.io/python-course/} \\
Revised versions of the notebooks will be made available on Nestor.

% Downey, A. (2015). \emph{Think Python. How to think like a computer scientist}.
% Needham, Massachusetts: Green Tea Press.
% Available online: \url{http://greenteapress.com/thinkpython2/html/index.html}
%
% Zelle, J. M. (2017).
% \emph{Python programming: an introduction to computer science}.
% Portland, Oregon: Franklin, Beedle \& Associates, Inc.
% (not freely downloadable; make sure to get the version for Python 3).
%
% Vierthaler (2018). Hacking the Humanities video tutorials.
% \url{https://www.youtube.com/playlist?list=PL6kqrM2i6BPIpEF5yHPNkYhjHm-FYWh17}


\section{Weekly schedule}\label{sectimetable}
\begin{enumerate}
\item Week 37, Sep 7--11; topics: Numbers, text, variables \\
    Read Severance ch.\ 1, 2.\\
    Lab: Notebook ch.\ 1.

\item Week 38, Sep 14--18; topics: Lists, conditions, and loops \\
    Read Severance ch.\ 3, 5, 8. \\
    Lab: Notebook ch.\ 2.

\item Week 39, Sep 21--25; topics: Functions, dictionaries, text processing \\
    Read Severance ch.\ 4, 6, 9. \\
    Lab: Notebook ch.\ 3.

\item Week 40, Sep 28--Oct 2; topics: Files, libraries \\
    Read Severance ch.\ 7. \\
    Lab: Notebook ch.\ 4. \\
    Midterm exam: individual assignment;
    deadline: Friday, October 2, 18:00;
    resit: Friday, November 6, 18:00.

\item Week 41, Oct 5--Oct 9; topics: Text analysis, visualization \\
    Lab: Notebook ch.\ 4

\item Week 42, Oct 12--16; topics: Data analysis and plotting \\
    Read \href{https://www.learndatasci.com/tutorials/python-pandas-tutorial-complete-introduction-for-beginners/}{Pandas tutorial}. \\
    Lab: Notebook ch.\ 5

\item Week 43, Oct 19--23; topics: More data analysis, course overview \\
    Lab: Notebook ch.\ 5. \\
    Final exam: individual assignment;
    deadline: Friday, October 30, 18:00;
    resit: Friday, January 15, 18:00.
\end{enumerate}

\section{Contact}

Andreas van Cranenburgh\\
\texttt{a.w.van.cranenburgh@rug.nl} \\
room 1311.411 \\
(However, for questions it is preferred to use the discussion board on Nestor).

\end{document}
