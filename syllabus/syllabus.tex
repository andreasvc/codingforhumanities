\documentclass[a4paper,12pt]{article}
\usepackage{hyperref,kpfonts,inconsolata,microtype,booktabs,enumitem,longtable,graphicx}
\hypersetup{colorlinks=true,urlcolor=blue}
\makeatletter\renewcommand\section{\@startsection {section}{1}{\z@}%
		    {-3.5ex \@plus -1ex \@minus -.2ex}%
		    {2.3ex \@plus.2ex}%
		    {\normalfont\large\scshape}}
\renewcommand\subsection{\@startsection{subsection}{2}{\z@}%
		{-3.25ex\@plus -1ex \@minus -.2ex}%
		{1.5ex \@plus .2ex}%
		{\normalfont\scshape}}
\renewcommand\thesection{\arabic{section} / }
\makeatother
\parindent 0pt
\parskip 7pt

\title{
    \includegraphics{rug}\\
    \textsc{Syllabus:} Coding for Humanities}
\author{Andreas van Cranenburgh}
\date{2024/2025}

\begin{document}
\maketitle
\thispagestyle{empty}
\pagestyle{empty}


\begin{tabular}{l p{0.75\textwidth} }
    Title:      & Coding for Humanities \\
    Code:       & LHU002M05 \\
    Programme:  & Master Communication and Information Studies (CIS) \\
    Tracks:     & Digital Humanities, Computer Communication \\
    Period:     & Semester 1a \\
    Type:       & Lectures and practical lab sessions \\
    Teacher:    & Andreas van Cranenburgh \\
    Credits:    & 5 ECTS \\
    Lectures:   & Tuesday 9:00-11:00, A902 \\
    Labs:       & Tuesday 15:00-17:00, MM room 2 \\
                %& Wednesday 17:00-19:00, MM room 1 \\
\end{tabular}



\section{Type of course unit, number of ects credit points,
        and admission requirements}
\begin{enumerate}[label={(\alph*)}]
    \item Type: mandatory (except for students CIS-Computer Communication
        starting in February).
    \item Credit points: 5 ECTS (140 hours).
    \item Admission requirements: (International) Students admitted to the
        tracks Digital Humanities (DH) or Computer Communication (CC) of the
        Master CIS are allowed to follow this course. No knowledge of
        programming or computer science is assumed. In case the student has
        already followed programming courses, it is wise to consult the
        teachers or study advisor to see whether the course is worthwhile to
        follow.
\end{enumerate}

\section{Content of the course unit}

The course is meant to be an introduction to programming in view of using
technology in solving problems in the interdisciplinary fields of Digital
Humanities (DH) and Computer-Mediated Communication (CMC). The basics of
programming are introduced, as well as tools and libraries
that are useful for DH and CMC. The programming language Python
(version 3) is used for writing programs. Jupyter is used as environment to
develop and execute Python programs. More specifically, the following basic
programming topics are addressed in the course: variables, values of various
types, strings, lists, dictionaries, conditionals, iteration and looping,
files, and functions. This is followed by practical applications related to
text analysis and processing of numerical data contained in tables.

\section{Position of the course unit in the degree program}
The course unit is accessible to master students of Communication and
Information Studies (tracks Digital Humanities and Computer Communication).
The course introduces programming concepts and techniques, and practices
programming skills which are necessary or useful for subsequent courses in the
programmes of Digital Humanities and Computer Communication.

\section{Learning outcomes of the course unit}

Upon successful completion of the course unit, students are able to:

\begin{enumerate}
    \item Write simple programs to perform basic tasks such as searching and
        cleaning text corpora
        (Application of Knowledge and Insight).
    \item Employ Jupyter Notebooks and other common Python data science
        tools to report on simple exploratory experiments: load a tabular
        dataset, compute summary statistics, and create plots
        (Application of Knowledge and Insight).
    \item Solve common errors during programming
        (Application of Knowledge and Insight).
    \item Evaluate documentation on available software to determine its
        applicability to a problem (Learning skills).
    \item Use proper terminology when collaborating with programmers
        (Communication).
\end{enumerate}

\section{Mode of instruction and learning activities}
In weekly lectures,
new programming concepts and techniques are introduced. Self-study is
required to get a better, deeper, and more nuanced understanding of the newly
introduced subject material. In the labs, students apply the knowledge obtained
in simple exercises, discussion, and programming assignments. Students have to
hand in two substantive and complex assignments (take-home exams), which will
be assessed (see Weekly Timetable in section 10). % ~\ref{sectimetable}).

Computers in the lab have the necessary software installed, but it may be
easier for you to use your own laptop. If you have one, please bring it to
the labs.

Attendance at the in person lectures and labs is encouraged since this allows for
the possibility to ask questions and other ways of interaction.

\section{Assessment}
The final course grade is based on an in-class exam in week 4 and a written
exam in the exam week.
Both exams have to be sufficient (at least 5.5) in order to
pass the course. The final score is the mean of the grades for the two exams,
weighted equally (50~\%). There is one re-sit opportunity for each exam, which
is available in case of an insufficient grade.
Dates for the exams will be announced on Brightspace.

\section{Cheating and plagiarism}
Cheating and plagiarism are subject to the provisions set down in the TER
(Article 8.17 of Part A of the BA TER or Article 4.13 of Part A of the MA TER).
The Board of Examiners is always informed in cases of suspected cheating or
plagiarism.

\section{Calculation of student workload}
\begin{tabular}{ll}
    Participation lectures, labs   & 35 hours \\
    Literature to be studied       & about 39 hours \\
    Assignments                    & about 50 hours \\
    (Preparing for) exams          & about 16 hours \\
    \emph{Total}                   & about 140 hours (5 ECTS) \\
\end{tabular}

\section{Literature}
Walsh, Melanie (2021). Introduction to Cultural Analytics \& Python.
    \url{https://melaniewalsh.github.io/Intro-Cultural-Analytics/}

\section{Weekly schedule}\label{sectimetable}
\begin{enumerate}
\item topics: Numbers, text, variables \\
    Lab: Notebook ch.\ 1.

\item topics: Lists, conditions, and loops \\
    Lab: Notebook ch.\ 2.

\item topics: Functions, dictionaries, text processing \\
    Lab: Notebook ch.\ 3.

\item topics: Files, libraries \\
    Lab: midterm exam, individual assignment \\
        start Notebook ch.\ 4.

\item topics: Text analysis, visualization \\
    Lab: Notebook ch.\ 4

\item topics: Data analysis and plotting \\
    Read \href{https://www.learndatasci.com/tutorials/python-pandas-tutorial-complete-introduction-for-beginners/}{Pandas tutorial}. \\
    Lab: Notebook ch.\ 5

\item topics: More data analysis, course overview \\
    Lab: Notebook ch.\ 5. \\
    Final exam: individual assignment
\end{enumerate}

\section{Contact}

Andreas van Cranenburgh\\
\texttt{a.w.van.cranenburgh@rug.nl} \\
room 1311.411 \\

\end{document}
