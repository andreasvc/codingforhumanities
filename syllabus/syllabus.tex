\documentclass[a4paper,12pt]{article}
\usepackage{hyperref,kpfonts,inconsolata,microtype,booktabs,enumitem,longtable,graphicx}
\hypersetup{colorlinks=true,urlcolor=blue}
\makeatletter\renewcommand\section{\@startsection {section}{1}{\z@}%
		    {-3.5ex \@plus -1ex \@minus -.2ex}%
		    {2.3ex \@plus.2ex}%
		    {\normalfont\large\scshape}}
\renewcommand\subsection{\@startsection{subsection}{2}{\z@}%
		{-3.25ex\@plus -1ex \@minus -.2ex}%
		{1.5ex \@plus .2ex}%
		{\normalfont\scshape}}
\renewcommand\thesection{\arabic{section} / }
\makeatother
\parindent 0pt
\parskip 7pt

\title{
    \includegraphics{rug}\\
    \textsc{Syllabus:} Coding for Humanities}
\author{Andreas van Cranenburgh, Leonie M. Bosveld-de Smet}
\date{2019/2020}

\begin{document}
\maketitle
\thispagestyle{empty}
\pagestyle{empty}


\begin{tabular}{l p{0.75\textwidth} }
    Title:      & Coding for Humanities \\
    Code:       & LHU002M05 \\
    Programme:  & Master Communication and Information Studies (CIS) \\
    Tracks:     & Digital Humanities, Computer Communication \\
    Period:     & Semester 1a \\
    Type:       & Hands-on Lectures and Practical Sessions \\
    Teachers:   & Andreas van Cranenburgh, \\
                & Leonie M. Bosveld-de Smet \\
    Credits:    & 5 ECTS \\
    Lectures:   & Monday 16:00-18:00, H1315.0036 \\
    Labs:       & Group 1: Thursday 15:00-18:00, MM1; \\
                & Group 2: Tuesday 15:00-18:00, MM2 \\
\end{tabular}



\section{Type of course unit, number of ects credit points,
        and admission requirements}
\begin{enumerate}[label={(\alph*)}]
    \item Type: mandatory (except for students CIS-Computer Communication
        starting in February).
    \item Credit points: 5 ECTS (140 hours).
    \item Admission requirements: (International) Students admitted to the
        tracks Digital Humanities (DH) or Computer Communication (CC) of the
        Master CIS are allowed to follow this course. No knowledge of
        programming or computer science is assumed. In case the student has
        already followed programming courses, it is wise to consult the
        teachers or study advisor to see whether the course is worthwhile to
        follow.
\end{enumerate}

\section{Content of the course unit}

The course is meant to be an introduction to programming in view of using
technology in solving problems in the interdisciplinary fields of Digital
Humanities (DH) and Computer-Mediated Communication (CMC). The very basics of
programming are introduced, as well as data structures, manipulations, and
modules that are useful for DH and CMC. The programming language Python
(version 3) is used for writing programs. Jupyter is used as environment to
develop and execute Python programs. More specifically, the following topics
are addressed in the course: variables, values of various types, strings,
lists, dictionaries, files, assignment, input and output statements,
conditionals, iteration and looping, functions, modules, libraries, regular
expressions, n-grams, processing data contained in a table.

\section{Position of the course unit in the degree program}
The course unit is accessible to master students of Communication and
Information Studies (tracks Digital Humanities and Computer Communication).
The course introduces programming concepts and techniques, and practices
programming skills which are necessary or useful for subsequent courses in the
programmes of Digital Humanities and Computer Communication.

\section{Learning outcomes of the course unit}

Upon successful completion of the course unit, students are able to:

% alternatieve versie:
% \begin{enumerate}
%     \item Write simple programs to perform basic tasks such as searching and
%         cleaning text corpora
%         (Application of Knowledge and Insight).
%     \item Work with Jupyter Notebooks and other common Python data science
%         tools to report on simple exploratory experiments: load a tabular
%         dataset, compute summary statistics, and create plots
%         (Application of Knowledge and Insight).
%     \item Understand and solve common errors during programming
%         (Application of Knowledge and Insight).
%     \item Read documentation on available software to evaluate its
%         applicability to a problem
%         (Learning skills).
%     \item Collaborate effectively with programmers using proper terminology
%         (Communication).
% \end{enumerate}


% versie Leonie:
\begin{enumerate}
    \item Describe, define, and illustrate basic programming concepts
        accurately, while making use of the appropriate programming
        terminology. (Knowledge and Insight)
    \item Solve small-scale problems involving numeric and textual data,
        devise an effective and efficient strategy to solve them,
        and implement the solution in Python code.
        (Application of Knowledge and Insight)
    \item Read, and understand Python programs, and add concise complementary
        descriptions and comments in natural language to it, in order to
        clarify the code, when code is not readable by itself. (Communication)
    \item Adequately use programming knowledge, insights, and skills obtained
        (i) to program in other programming languages similar to Python, and
        (ii) to address similar, but more complex, problems than the ones
        introduced in this introductory course. (Learning Skills)
\end{enumerate}


\section{Mode of instruction and learning activities}
The course is based on the principles of active learning. In weekly lectures,
new programming concepts and techniques are introduced. Self-study is
required to get a better, deeper, and more nuanced understanding of the newly
introduced subject material. In the labs, students apply the knowledge obtained
in simple exercises, discussion, and programming assignments. Students have to
hand in two substantive and complex assignments, which will be assessed (see
Weekly Timetable in section 10). % ~\ref{sectimetable}).

Computers in the lab have the necessary software installed, but it may be
easier for you to use your own laptop. If you have one, please bring it to
class.

\section{Assessment}
The course does not have a final exam. The final course grade is based on the
two assignments to hand in in week 4 and week 8. Both assignments have to be
sufficient (at least 5.5) in order to pass the course. The final score is based
on the grades for the two assignments, weighted equally (50~\%).
In case of an insufficient grade for either assignment, a resit possibility is
offered in week 10. Deadlines for handing in assignments are given in the
weekly timetable, and on Nestor.

\section{Cheating and plagiarism}
Cheating and plagiarism are subject to the provisions set down in the TER
(Article 8.17 of Part A of the BA TER or Article 4.13 of Part A of the MA TER).
The Board of Examiners is always informed in cases of suspected cheating or
plagiarism.

\section{Calculation of student workload}
\begin{tabular}{ll}
    Participation lectures, labs   & 35 hours \\
    Literature to be studied       & about 55 hours \\
    Assignments                    & about 50 hours \\
    \emph{Total}                   & about 140 hours (5 ECTS) \\
\end{tabular}

\section{Literature}
Karsdorp, F. and M. van Gompel. \emph{Python Programming for the Humanities}.
\url{http://www.karsdorp.io/python-course/})
An extended and updated version will be made available on Nestor.


\subsection*{Recommended background literature}

Downey, A. (2015). \emph{Think Python. How to think like a computer scientist}.
Needham, Massachusetts: Green Tea Press.
Available online: \url{http://-greenteapress.com/thinkpython2/}

Zelle, J. M. (2017).
\emph{Python programming: an introduction to computer science}.
Portland, Oregon: Franklin, Beedle \& Associates, Inc.
(not freely downloadable; make sure to get the version for Python 3).

Vierthaler (2018). Hacking the Humanities video tutorials.
\url{https://www.youtube.com/playlist?list=PL6kqrM2i6BPIpEF5yHPNkYhjHm-FYWh17}


\pagebreak
\section{Weekly schedule}\label{sectimetable}
{\raggedright
\begin{longtable}{p{0.18\linewidth} p{0.5\linewidth} p{0.3\linewidth} }
Week; date          & (Reading) assignments & Topics \\ \midrule

36; Sep 2--9 &
    Study Downey ch.\ 1, 2 (or Zelle sec.\ 2.1--2.5).
    Lab: Karsdorp \& van Gompel ch.\ 1 (getting started).
    & Numbers, Text, Variables
    \\

37; Sep 9--16     &
    Study Downey ch.\ 8, 10, 11;
    Zelle sec.\ 5.1-5.8, sec.\ 11.7;
    Karsdorp \& van Gompel ch.\ 1
    (string manipulation, lists,
    dictionaries).
    & Strings, Lists, Dictionaries
    \\

38; Sep 16--23    &
    Study Downey ch.\ 5;
    Zelle ch.\ 7, 8;
    Karsdorp \& van Gompel ch.\ 1 (conditions, loops).
    & Conditionals and Loops
    \\

39; Sep 23--30    &
    Study Downey ch.\ 3, 6, 14;
    Zelle ch.\ 5 (5.9), 6;
    Karsdorp \& van Gompel ch.\ 2.
    Midterm exam individual assignment to hand in;
    deadline: Friday, September 27, end
    of day; resit: Friday, November 8, end
    of day.
    & Functions, Files, Text clean-up
    \\

40; Sep 30--Oct 7 &
    Karsdorp \& van Gompel ch.\ 3.
    & Text Analysis, plotting
    \\

41; Oct 7--14     &
    Karsdorp \& van Gompel ch.\ 4.
    & Profiling, refactoring, debugging 
    \\

42; Oct 14--21    &
    Final exam: individual
    assignment to hand in; deadline:
    Friday, October 25, en of day; resit:
    Friday, November 8, end of day.
    & Data analysis and plotting with Pandas, Course Overview
    \\

\end{longtable}
}

\section{Contact}

Andreas van Cranenburgh\\
\texttt{a.w.van.cranenburgh@rug.nl} \\
room 1311.411


\end{document}
