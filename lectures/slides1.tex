\documentclass[aspectratio=169,usenames,dvipsnames]{beamer}
\usepackage{preamble}
\title{Coding for Humanities, week 1}
\begin{document}

\begin{frame}
 \titlepage
\end{frame}

\begin{frame}{Plan for today}
 \tableofcontents
\end{frame}




\section{Why programming}
\frame{\tableofcontents[currentsection]}
\begin{frame}{Programming: informal understanding}
    \begin{reference}
        Pine (2009). Learn to program. \url{https://pine.fm/LearnToProgram}
    \end{reference}
    \begin{definition}
        ``\structure{Programming} is telling your computer
        how to do something'' (Pine 2009, xii)
    \end{definition}
\end{frame}

\begin{frame}
    Why use computers in research?
	\pause

    \begin{itemize}
        \item Automation
        \item Amplification
        \item Reproducibility
    \end{itemize}

    \pause
    Computers need to be told what to do:

    \begin{itemize}
    \item Either user tools designed for a particular job
        \begin{itemize}
            \item but: limited to existing functionality
            \item cannot easily repeat a sequence of actions
        \end{itemize}

    \item \dots or create your own programs
        \begin{itemize}
            \item Higher investment to get started
            \item \dots but opens up lots of opportunities
        \end{itemize}
    \end{itemize}
\end{frame}

\begin{frame}{Research workflow}
    \begin{enumerate}
        \item Formulate research question
        \item Collect data
        \item Pre-process data
        \item Analyze data
        \item Draw conclusions
    \end{enumerate}

    \pause
    \dots programming is useful for steps 2--4
\end{frame}

\begin{frame}{Requirements for being a good programmer}
    \begin{itemize}
        \item Anyone can learn how to program
        \item Programming is a challenging activity:
            \begin{itemize}
                \item Analytic thinking
                \item Mental model of your program
                \item Paying attention to detail
            \end{itemize}
        \item Not everyone has the talent to become
            a \structure{good} programmer
    \end{itemize}
\end{frame}

\begin{frame}{Programming workflow in practice}
	\begin{columns}
		\column{0.3\linewidth}
			Many iterations of \dots
			\begin{itemize}
				\item Fail
				\item Fix
			\end{itemize}

		\column{0.7\linewidth}
			\includegraphics[width=0.45\linewidth]{fig/googling}
			\includegraphics[width=0.45\linewidth]{fig/stackoverflow}
	\end{columns}
\end{frame}



\subsection{Example DH projects}
\begin{frame}{Example DH projects}
\begin{reference}
\url{https://twitter.com/yaelnetzer/status/1158712777776730113}
\end{reference}
    \begin{columns}
        \column{0.5\linewidth}
            Student projects from a DH course by Yael Netzer (Ben Gurion univ\. Israel):
            \begin{itemize}
                \item Sentiment analysis of GoodReads reviews vs NY times reviews
                \item Game of Thrones character analysis
                \item How many black actors, directors,
                    producers are in movies?
                %\item Which terror attacks are reported in the NY times?
                \item Eurovision song festival voting by country
            \end{itemize}
        \column{0.5\linewidth}
            Required skills:
            \begin{description}
                \item[Text analysis] Get texts, extract/filter words, count
                \item[Data science] Get numerical data, explore, plot
            \end{description}
    \end{columns}
\end{frame}

\begin{frame}{Gender in Game of Thrones}\centering
    \includegraphics[width=0.6\textwidth]{fig/got}

    Words spoken for each gender across seasons
\end{frame}


\begin{frame}{Intermezzo: female computer pioneers}
	\begin{columns}
		\column{0.3\linewidth}
		Ada Lovelace %, mathematician

		\vspace{1ex}
		\includegraphics[width=0.9\linewidth,trim={0 1cm 0 1cm},clip]{fig/ada}
		
		Wrote \structure{first computer program}

		\pause
		\column{0.3\linewidth}
		Grace Hopper %, admiral, computer scientist
		
		\vspace{1ex}
		\includegraphics[width=0.9\linewidth]{fig/gracehopper}

		Invented machine-independent \structure{programming languages}

		\pause
		\column{0.3\linewidth}
		Margaret Hamilton %, computer scientist

		\vspace{1ex}
		\includegraphics[width=0.9\linewidth]{fig/margarethamilton}

		Lead programmer for \structure{Apollo Moon missions}
		% anecdote of finding bug
	\end{columns}
\end{frame}

\begin{frame}{Why Python}
    %\begin{definition}
    %   Python is a  ...
    %\end{definition}
    \begin{itemize}
        \item Easy and intuitive language:
            \begin{itemize}
                \item Code that is understandable as plain English
                \item Ease of development more important than fast programs
            \end{itemize}
        \item Suitable for many tasks:\\
            Many useful libraries (especially data science)
    \end{itemize}

    (Named after the TV show Monty Python)
\end{frame}

\begin{frame}
	% \begin{reference}
	% 	\href{https://www.economist.com/graphic-detail/2018/07/26/python-is-becoming-the-worlds-most-popular-coding-language}{
	% 		The Economist (2018): ``Python is becoming the world's most popular coding language''}
	% \end{reference}

    \centering
    \includegraphics[height=0.9\textheight]{fig/pythongrowth.png}

    Source: \url{https://stackoverflow.blog/2017/09/06/incredible-growth-python/}
\end{frame}

\begin{frame}[fragile]
A simple Python program:
\begin{lstlisting}[language=python]
print('Hello, World!")
\end{lstlisting}

\pause
\vspace{1em}
The same program in the Java programming language:

\begin{lstlisting}[language=java]
public class HelloWorld {
    public static void main(String[] args) {
        System.out.println("Hello, World!");
    }
}
\end{lstlisting}
\end{frame}


\begin{frame}
	\begin{columns}
		\column{0.5\linewidth}
			Traditional method:
			\begin{enumerate}
				\item edit file
				\item run
				\item repeat
			\end{enumerate}
		\column{0.5\linewidth}
			Notebooks
			\begin{itemize}
				\item Notebook: combines text, code, and results
				\item Graphical interface in browser
			\end{itemize}
			[demo of Jupyter notebook]
	\end{columns}
\end{frame}



\section{Programming fundamentals}
\subsection{Numbers, text, and variables}
\frame{\tableofcontents[currentsection]}

\begin{frame}[fragile]{Using Python as a calculator}
Operators:
    \begin{description}
        \item[Addition] 1 + 2
        \item[Subtraction] 2 - 1
        \item[Multiplication] 2 * 3
        \item[Division] 6 / 2
        \item[Exponentation] 3 ** 2
        \item[Parentheses] (1 + 2) * 3
    \end{description}
\end{frame}

\begin{frame}[fragile]{Using Python as a calculator}
One or more operators can be used to form an \structure{expression}
\begin{lstlisting}[language=python]
>>> 1 + 2 * 3
7
>>> (1 + 2) * 3
9
\end{lstlisting}

\pause
Note: just as in mathematics, multiplication/division takes precedence over
    addition/subtraction. Parentheses can be used to specify a different order.
\end{frame}

\begin{frame}[fragile]{Numeric data types}
    \begin{description}
        \item[Integer] whole numbers: 1, 2, 3, \dots -1, -2, -3
        \item[Floating point] numbers: 2.5, 1.3333, \dots
    \end{description}

\pause
\begin{lstlisting}[language=python]
>>> 5 / 2
2.5
>>> type(5 / 2)
<class 'float'>
>>> type(6 / 2)
<class 'int'>
\end{lstlisting}
\end{frame}

\begin{frame}[fragile]{Variables}
    \begin{definition}
        A \structure{variable} stores the result of an expression
        so that it can be re-used.
    \end{definition}

A variable is created/updated using `=':

\begin{definition}
\structure{Assignment}: \texttt{name = expression}

The name consists of one or more letters or underscores (case-sensitive).
\end{definition}

\begin{lstlisting}[language=python]
>>> a = 5 / 2
>>> a
2.5
\end{lstlisting}
\end{frame}

\begin{frame}[fragile]{Using variables}
Now, can use variable in expressions in place of a number:

\begin{lstlisting}[language=python]
>>> a + 0.5
3.0
\end{lstlisting}

\pause
A program is executed line-by-line, and variables can be overwritten:
\begin{lstlisting}[language=python]
>>> a = 2
>>> a
2
>>> a = 3
>>> a
3
\end{lstlisting}
\end{frame}


\begin{frame}[fragile]{Updating variables}
A common operation is to increase the value of a variable:
\begin{lstlisting}[language=python]
>>> a = 2
>>> a = a + 1
3
\end{lstlisting}

\pause
There is a shorthand for this operation:
\begin{lstlisting}[language=python]
>>> a = 2
>>> a += 1
3
\end{lstlisting}

Also -=, *=, etc.
\end{frame}


\begin{frame}{A practical example}
    Suppose I read War \& Peace in a year, \\
    what is my reading speed (words per minute)?
    \begin{itemize}
        \item Number of words in War \& Peace: 564,277
        \item Time spent reading per day: 1 hour
    \end{itemize}
    How to calculate average words per minute?
\end{frame}

\begin{frame}[fragile]{Solution}
\begin{lstlisting}[language=python]
>>> words_per_day = 564277 / 365
>>> words_per_min = words_per_day / 60
>>> words_per_min
25.766073059360732
\end{lstlisting}
\end{frame}


\begin{frame}[fragile]{Text}
Values can also be text instead of numbers:
\begin{lstlisting}[language=python]
>>> name = 'John'
>>> movie = '2001: A Space Odyssey'
\end{lstlisting}

\pause
    \begin{definition}
        A \structure{string}, short for string of characters,
        is a type of value that contains text.
    \end{definition}
\end{frame}

\begin{frame}{Types}
    We have seen numbers (integer, float) and text (string).

    These are examples of types:

    \begin{definition}
        A \structure{type} is a class of values recognized by the language
    \end{definition}
    
    \pause
    \begin{itemize}
        \item In Python, a value always has a type (a variable can be any type).
        \item The type defines what operations are valid.
    \end{itemize}
\end{frame}

\begin{frame}[fragile]{Different behavior for types}
    \begin{columns}
        \column{0.5\linewidth}
\begin{lstlisting}[language=python]
>>> 3 + 3
6
\end{lstlisting}

These are \structure{numbers}, so we can do arithmetic
        \column{0.5\linewidth}
\begin{lstlisting}[language=python]
>>> '3' + '3'
'33'
\end{lstlisting}

These are \structure{strings}, so they are treated as text
    \end{columns}
\end{frame}

\begin{frame}[fragile]{Invalid operations for types}
\begin{lstlisting}[language=python]
>>> 3 + '3'
Traceback (most recent call last):
  File "<stdin>", line 1, in <module>
TypeError: unsupported operand type(s) for +: 'int' and 'str'
\end{lstlisting}
    
    \centering
    \includegraphics[height=0.5\textheight]{fig/cantdothat}
\end{frame}
    

\begin{frame}[fragile]{Converting numbers to text and back}
\begin{lstlisting}[language=python]
>>> number = 42
>>> text = str(number)
>>> text
'42'

>>> int(text)
42
\end{lstlisting}

    \texttt{int(text)} is an example of using a function:

    \begin{definition}
        A \structure{function} is a piece of code that given arguments
        produces an effect or result.

        The notation f(x) means that the function f is \structure{applied}
        to the value x.
    \end{definition}

    \pause
    Here: \texttt{text} is given as argument to the function \texttt{int};
    a number is returned as a result.
\end{frame}


% \begin{frame}[fragile]{Converting numbers to text and back}
% \begin{lstlisting}[language=python]
% >>> name = 'John'
% >>> number = 7
% >>> name + ' has ' + str(number) + ' cousins'
% 'John has 7 cousins'
% \end{lstlisting}
% \end{frame}

\begin{frame}{Summary}
    \begin{itemize}
        \item Values and types:
            \begin{description}
                \item[Numbers:] int, float
                \item[Text:] str
            \end{description}
        \item Expressions: 1 + 2 * 3 - (4 + 5)
        \item Variables: a = 2
        \item Functions: str(2), int('2')
    \end{itemize}
\end{frame}


\subsection{Translate a formula into a Python program}
\begin{frame}{Example: computing readability}
	\begin{reference}
    Smith \& Senter (1968). Automated readability index
    \url{https://apps.dtic.mil/dtic/tr/fulltext/u2/667273.pdf}
	\end{reference}
    Automated readability index:
    A formula to estimate the difficulty of a text

    % \[
    %      0.50 ( \frac{\textsf{total words}}{\textsf{total sentences}} )
    %         + 4.71 ( \frac{\textsf{total characters}}{\textsf{total words}} )
    %         - 21.43
    % \]
    %
    % Result: a grade level
    \includegraphics[width=0.7\textwidth]{fig/ari}

    \pause
    \vspace{1em}
    \begin{itemize}
        \item multiplication is implicit in this notation: $ x(\dots) = $ x * (\dots)
        \item mistake in paper! s is both sentences and strokes!
    \end{itemize}
\end{frame}

\begin{frame}[fragile]
\begin{lstlisting}[language=python]
w = 500
s = 25
c = 3200

result = 0.5 * (w / s) + 4.71 * (c / w) - 21.43
\end{lstlisting}

\pause But why so cryptic?

\begin{itemize}
    \item Clear variable names
    \item Separate steps
\end{itemize}
\end{frame}

\begin{frame}[fragile]{Improved version}
\begin{lstlisting}[language=python]
total_words = 500
total_sentences = 25
total_characters = 3200


words_per_sent = total_words / total_sentences
chars_per_word = total_characters / total_words

result = 0.5 * words_per_sent + 4.71 * chars_per_word - 21.43
\end{lstlisting}
\end{frame}



\section{Course overview}
\frame{\tableofcontents[currentsection]}

\begin{frame}{Learning goals}
    \begin{itemize}
       \item 7 lectures \& labs
       \item 2 graded assignments \\
           (100\% of final grade)
    \end{itemize}
    After completing this course you will know the basics of \dots
    \begin{itemize}
       \item The Python programming language
       \item Text analysis
       \item Exploratory data analysis
       \item Fixing errors in programs
    \end{itemize}
\end{frame}


\begin{frame}{Course materials}
    \begin{itemize}
        \item Lectures with our material
        \item Lab: Notebooks from the course Python for the Humanities
        \item Preparation for lab:
            \begin{itemize}
                \item Video lectures from the course Hacking the Humanities
                \item Textbook: Think Python
            \end{itemize}
    \end{itemize}
\end{frame}

\begin{frame}{Course summary}
    \begin{itemize}
        \item \structure{Programming} is the creative and challenging process
            of writing a program
        \item A program is a sequence of unambiguous executable
            instructions to perform a task
        \item The code of a program is in some programming language
        \item Python is a high-level programming language
    \end{itemize}
\end{frame}

\begin{frame}{Some advice}
    \begin{itemize}
        \item This course gives a lot of (new) information
        \item Information is provided in a \structure{cumulative} way:
            each new piece of information builds on previous information
        \item Watch tutorials, read, and practice
        \item \structure{Ask questions} as soon as you think you are lost!
    \end{itemize}
\end{frame}

\begin{frame}{Code of conduct}
    \begin{itemize}
        \item Read the course manual (syllabus); available on Nestor
        \item On Nestor, under Course Documents, each week has:
            \begin{itemize}
                \item Lecture slides
                \item Lab exercises, discussion, and assignments
            \end{itemize}
        \item Grades will be on Nestor under My Gradebook
    \end{itemize}
\end{frame}

\begin{frame}{Code of conduct}
    \begin{itemize}
        \item \structure{Don't email} us on course content,
            programming issues, grading, etc.!
        \item For questions and remarks, address us face to face in class
            or make an appointment:
            \begin{itemize}
                \item Van Cranenburgh: office hours, room H1311, 411
                \item Bosveld: office hours, room H1311, 430
                \item Meijerhof; after lab hours, or make appointment
            \end{itemize}
    \end{itemize}
\end{frame}


\begin{frame}{Background reading}
    \begin{itemize}
        \item Downey ch.\ 1 and 2
        \item and/or Zelle sec.\ 2.1--2.5
            (make sure it is the latest edition for Python version 3).
        \item Watch youtube tutorials of "Hacking the Humanities" episode 1--5:
            \url{https://www.youtube.com/playlist?list=PL6kqrM2i6BPIpEF5yHPNkYhjHm-FYWh17}
    \end{itemize}
\end{frame}

\end{document}
