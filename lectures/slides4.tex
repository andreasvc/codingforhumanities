\documentclass[aspectratio=169,usenames,dvipsnames]{beamer}
\usepackage{preamble}
\title{Coding for Humanities, week 4}

\begin{document}

\begin{frame}
 \titlepage
\end{frame}

\begin{frame}{Last week}
    Dictionaries

    Functions

    Text (pre)processing
\end{frame}

\begin{frame}{Plan for today}
 \tableofcontents
\end{frame}

\section{Recap and last week's exercises}
\frame{\tableofcontents[currentsection]}

\subsection{Week 1: Arithmetic}
\begin{frame}[fragile]{Values}
    \begin{description}
        \item[Special] True, False, None
        \item[Numbers] -1, 2, 3.5, etc.
        \item[Strings] 'a', 'hello', "I'm"
    \end{description}
\end{frame}

\begin{frame}[fragile]{Variables}
    \begin{description}
        \item[Assignment] name = expression
        \item[Increment] number += 1
        \item[Expressions] number = 1 + 2 * other\_number + int('42')
    \end{description}
\end{frame}

\subsection{Week 2: Sequences, Ifs, Loops}
\begin{frame}[fragile]{Sequences}
    \begin{description}
        \item[Indexing/slicing] seq[0], seq[2:4]
        \item[Length] len(seq)
    \end{description}
\end{frame}
\begin{frame}[fragile]{Lists}
    \begin{description}
        \item[Lists in lists] the\_list[0][1]
        \item[Append] the\_list.append(item)
    \end{description}
\end{frame}
\begin{frame}[fragile]{Ifs}
    \begin{description}
        \item[Conditions] a == b, letter == 'a', number == 2
        \item[If] if condition: ...
    \end{description}
\end{frame}
\begin{frame}[fragile]{Loops}
    \begin{description}
        \item[For] for item in iterable: ...
    \end{description}
\end{frame}

\subsection{Week 3: Dictionaries and Functions}
\begin{frame}[fragile]{Dictionaries}
    \begin{description}
        \item[creation] example = {'a': 0, 'b': 1}
        \item[lookup] example['a']
    \end{description}
\end{frame}
\begin{frame}[fragile]{Functions}
    \begin{description}
        \item[Defining] def name(param1, ...): ...
        \item[Calling] name(arg1, ...)
        \item[Return value] return
    \end{description}
\end{frame}

% list comprehensions!

\begin{frame}[fragile]{extract\_dialogue}
\begin{columns}
\column{0.5\linewidth}
Split paragraphs
\begin{lstlisting}
with open(filename) as infile:
    text = infile.read()
paragraphs = text.split('\n\n')
\end{lstlisting}
\pause
Loop over paragraphs
\begin{lstlisting}
result = {}
for para in paragraphs:
\end{lstlisting}

\column{0.5\linewidth}
\pause
Identify lines with dialogue
\begin{lstlisting}
    first_token = para.split()[0]
    if first_token.endswith('.'):
\end{lstlisting}
\pause Split name and line
\begin{lstlisting}
        name, line = para.split('.', 1)
        # remove leading/trailing spaces
        name = name.strip()
\end{lstlisting}
Store
\begin{lstlisting}
        if name not in result:
            result[name] = []
        result[name].append(line)
\end{lstlisting}
\end{columns}
\end{frame}





\section{Working with files}
\frame{\tableofcontents[currentsection]}

\begin{frame}{Motivation}
    To do anything non-trivial,

    programs need to work on \structure{arbitrary data}

    \pause
    Therefore, need to learn how to \structure{read} external data
	from disk as input,

    and \structure{write} results as output

\end{frame}


\begin{frame}{Files and directories}
    \begin{itemize}
        \item Data is stored in \structure{files} as a sequence of bytes
        \item Files are part of \structure{directories} (folders)
        \item Directories may contain files and other directories. Ergo:
            \begin{itemize}
                \item The file system is a tree structure.
                \item The location of a file is a path through this tree.
            \end{itemize}
    \end{itemize}
\end{frame}

\begin{frame}{Paths}
    \begin{itemize}
        \item A path describes the location of a file
        \item In the simplest case, just the filename is enough to
            refer to a file in the same directory as your program:

            \texttt{example.txt}
        \item If a file is somewhere else, need to include directories:

            \texttt{data/example.txt}

    \end{itemize}
    \structure{NB}: Windows uses \textbackslash instead of \texttt{/}.
        However, this causes problems, use \texttt{/} in Python!
\end{frame}

\begin{frame}{Why directories}
    % Motivate organization
    \begin{itemize}
        \item Directories are useful to organize your files
        \item Use short, sensible names
        \item Don't put everything in one big pile. \\
            For example: separate code, data, results
    \end{itemize}
    \pause 
    Learn to use your system's file manager (Finder, Explorer, etc.)
    \begin{itemize}
        \item Make directories
        \item Move/rename files
        \item etc.
    \end{itemize}
    
\end{frame}

\begin{frame}[fragile]{Messy and organized files}
\begin{columns}
\column{0.5\linewidth}
\begin{verbatim}
Untitled 1.pdf
Untitled 1(1).pdf
Untitled 1(2).pdf
Untitled 1(2)(2)_draft.pdf
Untitled 1(2)_draftFINAL.pdf
Untitled 1(2)_FINAL_comments.pdf
Untitled 1(2)_FINAL_REALLY_FINAL.pdf
Untitled 1(2)_FINAL_make_it_stop.pdf
Untitled 1(2)_lAST-mInUtE-ChAnGES.pdf
\end{verbatim}
\pause
\column{0.5\linewidth}
\begin{verbatim}
code/ ...
data/ ...
research/ ...
teaching/ ...
└── codingforhumanities/
    ├── lectures/
    │   ├── slides1.pdf
    │   ├── slides2.pdf
    │   └── fig/ ...
    ├── notebooks/
    │   ├── ...
    │   └── data/ ...
    └── syllabus/
        └── syllabus.pdf
\end{verbatim}
\end{columns}
\end{frame}


\section{Importing modules}
\frame{\tableofcontents[currentsection]}

\begin{frame}[fragile]{Importing modules}
To go beyond builtin functionality, use import:
\begin{lstlisting}
# import whole module:
import <module>
# or import a specific part:
from <module> import <function>
\end{lstlisting}

\pause
\begin{lstlisting}
In: import math
In: math.pi
Out: 3.141592653589793
In: from math import sqrt
In: sqrt(16)
4.0
\end{lstlisting}

(Matter of preference/convention which you use)
\end{frame}

\subsection{The Python standard library}
\begin{frame}[fragile]{The Python standard library}
	"Batteries included"

	\vspace{1em}
	Many useful functions

	\vspace{1em}
    See \url{https://docs.python.org/3/library/index.html}
\end{frame}

\begin{frame}[fragile]{Finding files}
    \begin{itemize}
        \item The term ``globbing'' means ``matching a set of files with a search pattern.''
        \item The most common patterns are:
            \begin{itemize}
                \item \texttt{*} meaning ``match zero or more characters''
                \item \texttt{?} meaning ``match exactly one character''
            \end{itemize}
        \item Python contains the glob library to provide pattern matching functionality
        \item The glob library contains a function also called glob to match
            file patterns
        %\item e.g., \texttt{glob.glob('*.txt')} matches all files in the
        %    current directory whose names end with .txt.
    \end{itemize}
\pause
Example:
\begin{lstlisting}
In: from glob import glob
In: glob('Chapter*')
Out: ['Chapter 2 - Basic text processing.ipynb',
    'Chapter 1 - Getting started.ipynb',
    'Chapter 3 - Text analysis.ipynb',]
In: glob('data/*.txt')
Out: ['austen-emma.txt', 'world-domination.txt', 'evil-plans.txt']
\end{lstlisting}
\end{frame}

\begin{frame}[fragile]{Counter}
\begin{itemize}
\item A Counter is a specialized dictionary for counting
\item values are (integer) counts; missing values are 0
\item Mathematical term: \structure{bag}, \structure{multiset}
\end{itemize}

\begin{lstlisting}
In: from collections import Counter
In: cnt = Counter(['the', 'cat', 'is', 'on', 'the', 'mat'])
In: cnt
Out: Counter({'the': 2, 'is': 1, 'mat': 1, 'on': 1, 'cat': 1})
\end{lstlisting}
\end{frame}

\begin{frame}[fragile]{Counter: initialization}
Several ways to initialize a Counter:\\

\begin{lstlisting}
In: cnt = Counter()
In: cnt
Out: Counter()
In: cnt = Counter(['red', 'blue', 'red', 'blue', 'red'])
In: cnt
Out: Counter({'red': 3, 'blue': 2})
In: cnt = Counter({'red': 3, 'blue': 2})
In: cnt
Out: Counter({'red': 3, 'blue': 2})
\end{lstlisting}
\end{frame}

\begin{frame}[fragile]{Word frequency counting}
\begin{lstlisting}
words = [...]       # Some list of words
counts = {}
\end{lstlisting}\pause
\begin{lstlisting}
for word in words:
    if word not in counts::
        counts:[word] = 0
	counts:[word] += 1
...
\end{lstlisting}
\pause
Better:
\begin{lstlisting}
from collections import Counter
counts: = Counter(words)
\end{lstlisting}
\end{frame}

\begin{frame}[fragile]{Counter: more methods}
\begin{lstlisting}
In: cnt = Counter(['red','blue','red','blue','red'])
In: cnt.most_common(1)
Out: [('red', 3)]
In: for item, count in cnt.most_common():
...     print(count, item)
3 red
2 blue
In: cnt.update(['red', 'blue', 'blue'])
In: cnt
Out: Counter({'red': 4, 'blue': 4})

\end{lstlisting}
%In: Counter('abracadabra').most_common(3)
%Out: [('a', 5), ('r', 2), ('b', 2)]
\end{frame}


\subsection{The Python data science ecosystem}
\begin{frame}[fragile]{The Python data science ecosystem}
    \begin{description}
        \item[Spacy] industrial-strength Natural Language Processing (NLP)
        \item[NLTK] "educational-strength" NLP ...
        \item[Matplotlib] low-level plots/visualization
        \item[Seaborn] easier plot interface
        \item[Pandas] work with tabular data
        \item[Scikit-learn] Machine learning: classification, clustering
        \item[gensim] Topic modeling (LDA).
    \end{description}

	
\end{frame}

\begin{frame}{Installing packages}
    \begin{itemize}
        \item Most of these are part of Anaconda distribution (already installed!)

			(Exceptions: Spacy, gensim)
        \item Install conda packages with Anaconda Navigator \\
			(a kind of curated ``app store'')

            \url{https://docs.anaconda.com/anaconda/navigator/getting-started/\#navigator-managing-packages}

        \item Install other packages with \texttt{pip} \\
			(more advanced, much more packages)

            \url{https://www.puzzlr.org/install-packages-pip-conda-environment/}
    \end{itemize}
\end{frame}


\begin{frame}{Summary}
    \begin{itemize}
        \item Standard library
        \item Data science ecosystem
    \end{itemize}
\end{frame}

\end{document}
