\documentclass[aspectratio=169,usenames,dvipsnames]{beamer}
\usepackage{preamble}
\title{Coding for Humanities, week 4}

\begin{document}

\begin{frame}
 \titlepage
\end{frame}

\begin{frame}{Last week}
    Conditions

    Loops
\end{frame}

\begin{frame}{Plan for today}
 \tableofcontents
\end{frame}


\section{Defining functions}
\frame{\tableofcontents[currentsection]}

\begin{frame}{Motivation}
    \begin{itemize}
        \item We've used builtin functions (print, len, sorted);\\
            now we'll see how to make our own functions
        \item Why? Modularity; i.e., break down programs into smaller components
            \begin{itemize}
                \item Can create larger, more complex programs
                    from readily understandable building blocks
                \item Functions encapsulate complexity;
                    don't need to understand how a function works to use it
                \item Components are re-usable;
                    write once, use many times
            \end{itemize}
    \end{itemize}
\end{frame}

\begin{frame}[fragile]{Defining functions}
    Basic syntax:
\begin{lstlisting}
def function(param1, param2, ...):
    # do something with parameters
\end{lstlisting}

    A function definition has\dots
    \begin{itemize}
        \item A name
        \item A fixed number of parameters (may be zero)
        \item An indented block of code
    \end{itemize}
\end{frame}

\begin{frame}[fragile]{Defining a function does not run it}
    \begin{itemize}
        \item Executing a \texttt{def} statement does not run the code!
        \item The function is stored along with its name
        \item Function must be called to run it
    \end{itemize}
    Example:
\begin{lstlisting}
def greeting():
    print('Hello!')

greeting()
\end{lstlisting}
\end{frame}

\begin{frame}[fragile]{Function arguments}
    \begin{definition}
        A \structure{parameter} is a value expected by a function

        An \structure{argument} is a particular value for a parameter
            when calling a function
    \end{definition}

    \begin{itemize}
        \item Can pass data to functions using arguments
        \item Below, the argument \texttt{'John'} is assigned
            to the parameter \texttt{name} when the function is called
        \item Arguments have a specific order
        \item When calling a function, must use correct number of arguments
    \end{itemize}
Example:
\begin{lstlisting}
def greeting(name):
    print('Hello', name)

greeting('John')
\end{lstlisting}
\end{frame}

\begin{frame}[fragile]{Returning results}
    \begin{itemize}
        \item A function does not need to return a result \\
            In this case, 
            \begin{itemize}
                \item the function is used for its \emph{effect};\\
                    it does something (e.g., print text on screen)
                \item The value \texttt{None} is implicitly returned
            \end{itemize}
        \item Use \texttt{return ...} to give a result back to the caller
        \item May occur anywhere in the function
    \end{itemize}

Example:
\begin{lstlisting}
def average(values):
    if len(values) == 0:
        return None
    return sum(values) / len(values)
\end{lstlisting}
\end{frame}

\begin{frame}{Summary}
    \begin{itemize}
        \item Break programs down into functions to make them easier
            to understand.
        \item Define a function using \texttt{def} with a name, parameters,
            and a block of code.
        \item Defining a function does not run it.
        \item Arguments in call are matched to parameters in definition.
        \item Functions may return a result to their caller using return.
    \end{itemize}
\end{frame}




\section{Cleaning text}
\frame{\tableofcontents[currentsection]}
\begin{frame}{Goal: text $\rightarrow$ words}
    For basic text analysis,

    want to consider only word frequencies of a text

    \pause
    Therefore:
    \begin{enumerate}
        \item Identify words
        \item Simplify text, remove non-words
        \item Count words to get frequencies
    \end{enumerate}
\end{frame}

\begin{frame}[fragile]{Upper and lower case}
    \begin{itemize}
        \item To the computer, these
            are three completely different words!

            \texttt{'John', 'john', 'JOHN'}
        \item Often, we don't want to make this distinction when counting
        \item One solution: case folding
    \end{itemize}
    \pause
\begin{lstlisting}
text = 'HeLlO WoRlD'
cleaned = text.lower()
\end{lstlisting}
\end{frame}

\begin{frame}[fragile]{Punctuation}
    \begin{itemize}
        \item Similarly, punctuation should be separated from words
        \item Or should be eliminated completely
        \item Can use functionality similar to "find \& replace" for this
    \end{itemize}
    \pause
\begin{lstlisting}
text = "Forsooth, 'tis true!"
text = text.replace('!', '')
text = text.replace(',', '')
\end{lstlisting}
\end{frame}


\begin{frame}{Summary}
\end{frame}

% parsing a simple text file line-by-line, e.g. subtitles?
\end{document}
