\documentclass[aspectratio=169,usenames,dvipsnames]{beamer}
\usepackage{preamble}
\title{Coding for Humanities, week 5}

\begin{document}

\begin{frame}
 \titlepage
\end{frame}

\begin{frame}{Last week}
    Conditions

    Loops
\end{frame}

\begin{frame}{Plan for today}
 \tableofcontents
\end{frame}


\section{Working with files}
\frame{\tableofcontents[currentsection]}

\begin{frame}{Motivation}
    To do anything non-trivial,

    programs need to work on \structure{arbitrary data}

    \pause
    Therefore, need to learn how to \structure{read} data from disk as input,

    and \structure{write} results as output

\end{frame}


\begin{frame}{Files and directories}
    \begin{itemize}
        \item Data is stored in \structure{files}
        \item Files are part of \structure{directories} (folders)
        \item Directories may contain files and other directories
    \end{itemize}
\end{frame}

\begin{frame}{Paths}
    \begin{itemize}
        \item A path describes the location of a file
        \item In the simplest case, just the filename is enough to
            refer to a file in the same directory as your program:

            \texttt{example.txt}
        \item If a file is somewhere else, need to include directories:

            \texttt{corpus/example.txt}
    \end{itemize}
\end{frame}

\begin{frame}{Why directories}
    % Motivate organization
    \begin{itemize}
        \item Directories are useful to organize your files
        \item Use sensible names
        \item Separate code, data, results
    \end{itemize}
\end{frame}

\section{Working with libraries}
\frame{\tableofcontents[currentsection]}

\begin{frame}[fragile]{Aside: the Python standard library}
\begin{lstlisting}
import <module>
\end{lstlisting}

    See \url{https://docs.python.org/3/library/index.html}
\end{frame}



\begin{frame}[fragile]{Finding files}
    \begin{itemize}
        \item The term ``globbing'' means ``matching a set of files with a pattern.''
        \item The most common patterns are:
            \begin{itemize}
                \item \texttt{*} meaning ``match zero or more characters''
                \item \texttt{?} meaning ``match exactly one character''
            \end{itemize}
        \item Python contains the glob library to provide pattern matching functionality
        \item The glob library contains a function also called glob to match
            file patterns
        \item e.g., \texttt{glob.glob('*.txt')} matches all files in the
            current directory whose names end with .txt.
        \item Result is a (possibly empty) list of strings.
    \end{itemize}
\end{frame}

\begin{frame}{Summary}
\end{frame}

\end{document}
